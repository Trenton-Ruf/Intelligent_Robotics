%%%%%%%%%%%%%%%%%%%%%%%%%%%%%%%%%%%%%%%%%
% Wenneker Assignment
% LaTeX Template
% Version 2.0 (12/1/2019)
%
% This template originates from:
% http://www.LaTeXTemplates.com
%
% Authors:
% Vel (vel@LaTeXTemplates.com)
% Frits Wenneker
%
% License:
% CC BY-NC-SA 3.0 (http://creativecommons.org/licenses/by-nc-sa/3.0/)
% 
%%%%%%%%%%%%%%%%%%%%%%%%%%%%%%%%%%%%%%%%%

%----------------------------------------------------------------------------------------
%	PACKAGES AND OTHER DOCUMENT CONFIGURATIONS
%----------------------------------------------------------------------------------------

\documentclass[11pt]{scrartcl} % Font size

\input{structure.tex} % Include the file specifying the document structure and custom commands

%----------------------------------------------------------------------------------------
%	TITLE SECTION
%----------------------------------------------------------------------------------------

\title{	
	\normalfont\normalsize
	\textsc{Intelligent Robotics, Portland State University}\\ % Your university, school and/or department name(s)
	\vspace{25pt} % Whitespace
	\rule{\linewidth}{0.5pt}\\ % Thin top horizontal rule
	\vspace{20pt} % Whitespace
	{\huge HW1}\\ % The assignment title
	\vspace{4pt} % Whitespace
	{\large Fuzzy-PID Altitude Hold}\\ % The assignment title
	\vspace{12pt} % Whitespace
	\rule{\linewidth}{2pt}\\ % Thick bottom horizontal rule
	\vspace{12pt} % Whitespace
}

\author{\LARGE Trenton Ruf} % Your name
\date{\normalsize \today} % Today's date (\today) or a custom date

\begin{document}
\maketitle % Print the title
%\renewcommand\thesubsection{\Alph{subsection}}
%\renewcommand\thesubsection{\Roman{subsection}}
%\doublespacing
%\singlespace
%\onehalfspacing
%\setstretch{1.25}

%\begin{doublespace}
%\end{doublespace}

%\vspace*{\fill}
%\clearpage
%\vspace*{\fill}


%\vspace*{\fill}
\renewcommand\thesubsection{\Roman{subsection}}
%\section{Loading Pre-Trained Model}
\section{ROSflight Intro}

ROSflight is open source autopilot software that runs on a small embedded flight controller. A companion computer can be used to recieve sensor data from the flight controller and send back commands. Since ROS integrates well with the Gazebo physics simulator I am testing plane controls in simulations only this term. This is mainly so I can safely "crash" the plane a lot while I learn.

\section{Setup}
\subsection{Operating System}
I HIGHLY recomend using Ubuntu for setting up ROSflight.
The latest supported version of ROSflight requires ROS Melodic on Ubuntu 18.04. Which is a bit unfortunate since 18.04 is already out of LTS. Though there is work on getting it ready for ROS2. Some libraries I used such as Scikit-Learn's fuzzy controller requires manual installation of a later Python version that is not supported by 18.04.

\subsection{Roslaunch}
I modified the default Roslaunch file for simulating a fixed-wing aircraft. I added the companion node roslaunch_joy and bound it to my keyboard so I can simulate a Radio Controller to manually manuever the aircraft. The RC controller is necissary to "arm" the plane before the throttle can be activated.



%\clearpage

\end{document}
